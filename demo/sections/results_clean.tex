\section{Results}

This section presents the results of our multi-file LaTeX compilation test.

\subsection{File Structure Analysis}

The demo project successfully demonstrates the following multi-file capabilities:

\begin{itemize}
\item \textbf{File Inclusion}: Multiple \texttt{.tex} files are properly combined
\item \textbf{Cross-Referencing}: References work across file boundaries
\item \textbf{Table of Contents}: Automatically generated from all sections
\item \textbf{Page Numbering}: Continuous across all included files
\end{itemize}

\subsection{Compilation Success}

The LaTeX compilation should succeed if all files are present and properly structured. Refer to Table~\ref{tab:comparison} for a comparison of different approaches.

\subsection{Mathematical Results}

Additional mathematical expressions to verify compilation:

\begin{equation}
a^2 + b^2 = c^2
\end{equation}

This is the famous Pythagorean theorem relating the sides of a right triangle.

\subsection{List Examples}

Different types of lists to test formatting:

\begin{enumerate}
\item First level item
  \begin{enumerate}
  \item Second level item
    \begin{enumerate}
    \item Third level item
    \end{enumerate}
  \end{enumerate}
\item Another first level item
\end{enumerate}

\subsection{Code and Algorithms}

Here's a sample algorithm:

\begin{verbatim}
function compileLatex(files) {
  for (let file of files) {
    if (file.includes('\include{')) {
      processIncludedFiles(file);
    }
  }
  generatePDF();
}
\end{verbatim}

This demonstrates that verbatim environments work correctly across file boundaries.