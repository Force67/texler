\section{Conclusion}

This multi-file LaTeX demo project successfully demonstrates the capabilities of Texler's compilation system.

\subsection{Summary of Achievements}

We have successfully created and tested:

\begin{enumerate}
\item A main document that includes multiple section files
\item Proper LaTeX structure with cross-file references
\item Mathematical expressions and environments
\item Tables, lists, and formatted content
\item Table of contents generation
\end{enumerate}

\subsection{Technical Validation}

The compilation process validates that:
\begin{itemize}
\item The Docker container can handle multiple LaTeX files
\item File paths are correctly resolved during compilation
\item All standard LaTeX features work in multi-file context
\item Error handling maintains context across files
\end{itemize}

\subsection{Next Steps for Texler}

Based on this successful test, the next steps for implementing multi-file support in Texler could include:

\begin{description}
\item[UI Enhancement] Add file browser and tab system
\item[State Management] Extend from single-file to multi-file state
\item[User Experience] Provide seamless file switching and navigation
\item[Performance] Optimize compilation for large projects
\end{description}

\subsection{Final Remarks}

This demo project serves as a foundation for implementing comprehensive multi-file LaTeX editing capabilities in Texler. The backend infrastructure is already capable -- the focus should now be on creating an intuitive user interface for managing multiple LaTeX files.

\vspace{1cm}
\begin{center}
\textit{This concludes our multi-file LaTeX demonstration.}
\end{center}