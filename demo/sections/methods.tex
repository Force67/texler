\section{Methods}

This section describes the methodology used in this demonstration.

\subsection{File Organization}

The project is organized as follows:

\begin{description}
\item[main.tex] The main document file that includes all sections
\item[sections/] Directory containing individual section files
\item[introduction.tex] Introduction and background
\item[methods.tex] This methods section
\item[results.tex] Results section
\item[conclusion.tex] Conclusion section
\end{description}

\subsection{LaTeX Commands Used}

We use several important LaTeX commands for multi-file projects:

\begin{itemize}
\item \texttt{\textbackslash include\{filename\}} - Includes a file and starts on a new page
\item \texttt{\textbackslash input\{filename\}} - Includes a file without page break
\item \texttt{\textbackslash tableofcontents} - Generates table of contents
\end{itemize}

\subsection{Compilation Process}

The compilation process for multi-file projects:

\begin{enumerate}
\item LaTeX reads the main file
\item When encountering \texttt{\textbackslash include}, it processes the included file
\item All cross-references and citations are resolved
\item Table of contents is generated
\item Final PDF is produced
\end{enumerate}

\subsection{Tables and Figures}

Here's a sample table to test formatting:

\begin{table}[h]
\centering
\begin{tabular}{|l|c|r|}
\hline
\textbf{Feature} & \textbf{Single File} & \textbf{Multi File} \\
\hline
Organization & Basic & Excellent \\
Compilation Speed & Fast & Slower \\
Maintainability & Poor & Good \\
Collaboration & Difficult & Easy \\
\hline
\end{tabular}
\caption{Comparison of single-file vs multi-file LaTeX projects}
\label{tab:comparison}
\end{table}